\documentclass{article}

\usepackage{amsmath, amsthm, amssymb, amsfonts}
\usepackage{thmtools}
\usepackage{graphicx}
\usepackage{setspace}
\usepackage{geometry}
\usepackage{float}
\usepackage{hyperref}
\usepackage[utf8]{inputenc}
\usepackage[english]{babel}
\usepackage{framed}
\usepackage[dvipsnames]{xcolor}
\usepackage{tcolorbox}
\usepackage{enumerate}   


\colorlet{LightGray}{White!90!Periwinkle}
\colorlet{LightBlue}{Blue!10}
\colorlet{LightOrange}{Orange!15}
\colorlet{LightGreen}{Green!15}

\newcommand{\HRule}[1]{\rule{\linewidth}{#1}}

\declaretheoremstyle[name=Theorem,]{thmsty}
\declaretheorem[style=thmsty,numberwithin=section]{theorem}
\tcolorboxenvironment{theorem}{colback=LightGreen}

\declaretheoremstyle[name=Definition,]{prosty}
\declaretheorem[style=prosty,numberwithin=section]{definition}
\tcolorboxenvironment{definition}{colback=LightBlue}

\declaretheoremstyle[name=Principle,]{prcpsty}
\declaretheorem[style=prcpsty,numberwithin=section]{principle}
\tcolorboxenvironment{principle}{colback=LightGreen}


\setstretch{1.2}
\geometry{
    textheight=9in,
    textwidth=5.5in,
    top=1in,
    headheight=12pt,
    headsep=25pt,
    footskip=30pt
}

% ------------------------------------------------------------------------------

\begin{document}

% ------------------------------------------------------------------------------
% Cover Page and ToC
% ------------------------------------------------------------------------------

\title{ \normalsize \textsc{}
		\\ [2.0cm]
		\HRule{1.5pt} \\
		\LARGE \textbf{\uppercase{Math 424}
		\HRule{2.0pt} \\ [0.6cm] \LARGE{} \vspace*{10\baselineskip}}
		}
\date{}
\author{\textbf{Author} \\ 
		Terry Qiu \\
		Urbana-Champaign \\
		Fall 2025}

\maketitle
\newpage

\tableofcontents
\newpage

% ------------------------------------------------------------------------------
\section{Metric Spaces}
\subsection{Definition of a metric space}
\begin{definition}
    A metric space $(M, d)$ is a set $M$ equipped with a binary operation $d:M\times M\rightarrow [0,\infty)$
    such that
    \begin{enumerate}[(i)]
        \item $d(p,q) \iff p=q $
        \item $d(p,q)=d(q,p)$
        \item $d(p,q) \leq d(p,r)+d(r,q)$
    \end{enumerate}
    for all $p,q,r\in M$.
\end{definition}
\begin{theorem}
    For points $p_1,p_2,...,p_n$ in a metric space.
    \[
    d(p_1, p_n) \leq d(p_1, p_2)+d(p_2,p_3)+...+d(p_{n-1},p_n)
    \]
\end{theorem}
\begin{proof}
    Notice that for $n=2$, $d(p_1,p_2)\leq d(p_1,p_2)$. Notice also that for $n=3$,
    $d(p_1,p_3)\leq d(p_1,p_2)+ d(p_2,p_3)$.
    These are our base cases.\\\\
    Suppose that the identity is true for $n\leq k-1$, we want to show that it is true for
    $n=k$. Notice that by the triangle inequality
    \[
    d(p_1,p_k)\leq d(p_1,p_2)+ d(p_2,p_k)
    \]
    By induction 
     \[
    d(p_1,p_k)\leq d(p_1,p_2)+ d(p_2,p_k)\leq d(p_1,p_2)+ d(p_2,p_3) +...+d(p_{k-1},p_k)
    \]
\end{proof}
\begin{theorem}
    If $p,q,r$ are points in a metric space then 
    \[
    |d(p,r)-d(q,r)| \leq d(p,q)
    \]
\end{theorem}

\subsection{Open and closed sets}
\begin{definition}
    Let $M$ be a metric space and let $p_0$ be an element of $E$. The set
    \[
    U(p_0,r) = \{p\in M| d(p_0, p) < r\}
    \]
    is called the open ball with radius $r$ centered at $p_0$. The closed ball centered at $p_0$ 
    with radius $r$ is the set
    \[
    U(p_0,r) = \{p\in M| d(p_0, p) \leq r\}
    \]

\end{definition}

\begin{definition}
    A subset $S$ of a metric space is an \textbf{open} set if for every point $s$ in $S$ there exists an open ball $U$ centered at that point such that 
    $U\subseteq S$.
\end{definition}
The following is true for open subsets
\begin{theorem}
    A finite intersection of open subsets is open
\end{theorem}
\begin{proof}
    Let $S_1,S_2,...,S_n$ be open subsets of a metric space $M$. Let \[ S=\bigcap_{i=1,2,...,n}S_i\] be the intersection. Take any element $p$ of the intersection. 
    Since it is in each $S$, there exists an open ball centered at $p$ with radius $r_i$ for each $S_i$. Let $r=\min{(r_1,r_2,...,r_n)}$. Without loss of generality, suppose that
    $r_1 \leq r_2\leq...\leq r_n$, then by definition $U(p,r) =U(p,r_1) \subseteq U(p, r_2) \subseteq ... \subseteq U(p, r_n)$. But then each of these balls are contained within their respective open
    subsets $S_i$. So the open ball centered at $p$ with radius $r$ is contained within each of $S_i$ so it is contained in the intersection of all $S_i$.
\end{proof}
\begin{theorem}
    A open ball is an open set
\end{theorem}
\begin{definition}
    A subset $S$ of a metric space is closed if its complement is open.
\end{definition}
\begin{theorem}
    A closed ball is a closed set
\end{theorem}



% ------------------------------------------------------------------------------
% Reference and Cited Works
% ------------------------------------------------------------------------------

\bibliographystyle{IEEEtran}
\bibliography{References.bib}

% ------------------------------------------------------------------------------

\end{document}