\documentclass{article}

\usepackage{amsmath, amsthm, amssymb, amsfonts}
\usepackage{thmtools}
\usepackage{graphicx}
\usepackage{setspace}
\usepackage{geometry}
\usepackage{float}
\usepackage{hyperref}
\usepackage[utf8]{inputenc}
\usepackage[english]{babel}
\usepackage{framed}
\usepackage[dvipsnames]{xcolor}
\usepackage{tcolorbox}
\usepackage{enumerate}   


\colorlet{LightGray}{White!90!Periwinkle}
\colorlet{LightBlue}{Blue!10}
\colorlet{LightOrange}{Orange!15}
\colorlet{LightGreen}{Green!15}

\newcommand{\HRule}[1]{\rule{\linewidth}{#1}}

\declaretheoremstyle[name=Theorem,]{thmsty}
\declaretheorem[style=thmsty,numberwithin=section]{theorem}
\tcolorboxenvironment{theorem}{colback=LightGreen}

\declaretheoremstyle[name=Definition,]{prosty}
\declaretheorem[style=prosty,numberwithin=section]{definition}
\tcolorboxenvironment{definition}{colback=LightBlue}

\declaretheoremstyle[name=Principle,]{prcpsty}
\declaretheorem[style=prcpsty,numberwithin=section]{principle}
\tcolorboxenvironment{principle}{colback=LightGreen}


\setstretch{1.2}
\geometry{
    textheight=9in,
    textwidth=5.5in,
    top=1in,
    headheight=12pt,
    headsep=25pt,
    footskip=30pt
}

% ------------------------------------------------------------------------------

\begin{document}

% ------------------------------------------------------------------------------
% Cover Page and ToC
% ------------------------------------------------------------------------------

\title{ \normalsize \textsc{}
		\\ [2.0cm]
		\HRule{1.5pt} \\
		\LARGE \textbf{\uppercase{Math 424}
		\HRule{2.0pt} \\ [0.6cm] \LARGE{} \vspace*{10\baselineskip}}
		}
\date{}
\author{\textbf{Author} \\ 
		Terry Qiu \\
		Urbana-Champaign \\
		Fall 2025}

\maketitle
\newpage

\tableofcontents
\newpage

% ------------------------------------------------------------------------------
\section{Metric Spaces}
\subsection{Definition of a metric space}
\begin{definition}
    A metric space $(M, d)$ is a set $M$ equipped with a binary operation $d:M\times M\rightarrow [0,\infty)$
    such that
    \begin{enumerate}[(i)]
        \item $d(p,q) \iff p=q $
        \item $d(p,q)=d(q,p)$
        \item $d(p,q) \leq d(p,r)+d(r,q)$
    \end{enumerate}
    for all $p,q,r\in M$.
\end{definition}
\begin{theorem}
    For points $p_1,p_2,...,p_n$ in a metric space.
    \[
    d(p_1, p_n) \leq d(p_1, p_2)+d(p_2,p_3)+...+d(p_{n-1},p_n)
    \]
\end{theorem}
\begin{proof}
    Notice that for $n=2$, $d(p_1,p_2)\leq d(p_1,p_2)$. Notice also that for $n=3$,
    $d(p_1,p_3)\leq d(p_1,p_2)+ d(p_2,p_3)$.
    These are our base cases.\\\\
    Suppose that the identity is true for $n\leq k-1$, we want to show that it is true for
    $n=k$. Notice that by the triangle inequality
    \[
    d(p_1,p_k)\leq d(p_1,p_2)+ d(p_2,p_k)
    \]
    By induction 
     \[
    d(p_1,p_k)\leq d(p_1,p_2)+ d(p_2,p_k)\leq d(p_1,p_2)+ d(p_2,p_3) +...+d(p_{k-1},p_k)
    \]
\end{proof}
\begin{theorem}
    If $p,q,r$ are points in a metric space then 
    \[
    |d(p,r)-d(q,r)| \leq d(p,q)
    \]
\end{theorem}

\subsection{Open and closed sets}
\begin{definition}
    Let $M$ be a metric space and let $p_0$ be an element of $E$. The set
    \[
    U(p_0,r) = \{p\in M| d(p_0, p) < r\}
    \]
    is called the open ball with radius $r$ centered at $p_0$. The closed ball centered at $p_0$ 
    with radius $r$ is the set
    \[
    U(p_0,r) = \{p\in M| d(p_0, p) \leq r\}
    \]

\end{definition}

\begin{definition}
    A subset $S$ of a metric space is an \textbf{open} set if for every point $s$ in $S$ there exists an open ball $U$ centered at that point such that 
    $U\subseteq S$.
\end{definition}
The following is true for open subsets
\begin{theorem}
    A finite intersection of open subsets is open
\end{theorem}
\begin{proof}
    Let $S_1,S_2,...,S_n$ be open subsets of a metric space $M$. Let \[ S=\bigcap_{i=1,2,...,n}S_i\] be the intersection. Take any element $p$ of the intersection. 
    Since it is in each $S$, there exists an open ball centered at $p$ with radius $r_i$ for each $S_i$. Let $r=\min{(r_1,r_2,...,r_n)}$. Without loss of generality, suppose that
    $r_1 \leq r_2\leq...\leq r_n$, then by definition $U(p,r) =U(p,r_1) \subseteq U(p, r_2) \subseteq ... \subseteq U(p, r_n)$. But then each of these balls are contained within their respective open
    subsets $S_i$. So the open ball centered at $p$ with radius $r$ is contained within each of $S_i$ so it is contained in the intersection of all $S_i$.
\end{proof}
\begin{theorem}
    A open ball is an open set
\end{theorem}
\begin{definition}
    A subset $S$ of a metric space is closed if its complement is open.
\end{definition}
\begin{theorem}
    A closed ball is a closed set
\end{theorem}
\newpage
\subsection{Convergent Sequences}
\begin{definition}
    We say that a sequence of points $(x_n)$ is converges to $p$ if for every $\epsilon>0$, we can
    find an $N$ such that for all $n> N$, we have 
    \[
    d(x_n,p)<\epsilon
    \]
\end{definition}
In other words, for a convergent sequence there always exists a $N$-tail such that all terms in that $N$-tail are epsilon
close to $p$.
\begin{theorem}
    In a metric space $(X,d)$, every convergent has only one limit.
\end{theorem}
\begin{proof}
    Suppose there exists two limits $p_0$ and $p_1$ for some convergent sequence $(x_n)$. Then for each $\epsilon >0$,
    there exists $N,M$ such that $d(x_m, p_0)< \epsilon$, and $d(x_n, p_0)< \epsilon$ for all $n>N$ and $m>M$. Let $N'=\max(N,M)$,
    then for each $n>N'$. $d(p_0, p_1)\leq d(p_0, x_n)+d(x_n, p_1)< \epsilon + \epsilon= 2\epsilon$. But if $d(p_0,p_1)$ is not $0$, 
    there is a contradiction for all $\epsilon \leq \frac{d(p_0,p_1)}{2}$. So $d(p_0,p_1)=0$, in other words $p_0=p_1$.
\end{proof}
\begin{theorem}
    Any infinite subsequence of a convergent sequence converges to the same limit.
\end{theorem}
\begin{proof}
    Suppose $(x_n)$ is a convergent sequence with limit $p$. Then for each $\epsilon>0$, there exists an $N$ such that for all $n>N$,
    $d(x_n,p)<\epsilon$. Suppose that $n_1,n_2,n_3,...$is a strictly increasing sequence and that $(x_{n_m})$ defines a subsequence of $(x_n)$. For each $\epsilon>0$, since we can find an $N$ such that the $N$
    tail of $(x_n)$ is $\epsilon$ close to $p$, it is also the case that for each $n_m>N$, that $d(x_{n_m}, p)<\epsilon$. But this is precisely what it means for 
    $(x_{n_m})$ to be convergent. For any $\epsilon >0$ we can find an $N$ such that all terms in $(x_{n_m})$ after $N$ is $\epsilon$ close to $p$
\end{proof}
\begin{definition}
    A subset $S$ of a metric space is bounded if for every $s,t$ in $S$, there exists an $r>0$ such that $d(s,t)<r$. In other words, all elements 
    are contained within some open ball, so that no two elements are more than $r$ away from each other.
\end{definition}
\begin{definition}
    We say that a sequence $(x_n)$ is bounded if the set containing its terms $\left\{ x_1,x_2,x_3,... \right\}$ is bounded
\end{definition}
\begin{theorem}
    If the elements of a subset $S$ of a metric space is contained within an open ball of radius $r$ centered at some point $p$, then it is bounded
\end{theorem}
\begin{proof}
    Notice that it is enough to show that $d(s_1,s_2)<k$ between any two element $s_1,s_2$ for some $k>0$. Since $d(s,p)<r$ for any $s\in S$, any two terms in the sequence say $s_1, s_2$ satisfy $d(s_1,s_2)\leq d(s_1,p) +d(p,s_1)<2r$.
    So we have found a $k$ such that any two elements is no more than $k$ away. This is precisely what it means for a set to be bounded. 
\end{proof}
\begin{theorem}
    Every convergent sequence is bounded.
\end{theorem}
\begin{proof}
    Let $(x_n)$ be convergent sequence that converges to $p$. Then pick some $\epsilon >0$. By definition, there exists an $N$ such that for all $n>N$,
    $d(x_n, p)< \epsilon$. So every element of $\left\{ x_n,x_{n+1},\dots \right\}$ is contained within some open ball centered at $p$. To encompass the terms before $N$, let the 
    radius of some open ball be 
    \[
    r=\max\{ \epsilon, d(x_1,p),..., d(x_N,p)\}+1
    \]
    Then the set containing every term of the sequence is contained within an open ball of radius $r>0$, by theorem $1.9$, this means that the set is bounded. By definition, if the set containing
    the terms of the sequence is bounded, the sequence is bounded.
\end{proof}
\begin{theorem}
    A set $S$ is closed if and only if whenever $p_1,p_2,p_3,...$ is a convergent sequence of points in $S$, then the limit of that sequence is in $S$.
\end{theorem}
\begin{proof}
    Suppose that $S$ is closed and consider a convergent sequence $(p_n)$
\end{proof}

In other words, a subset $S$ of a metric space is closed if any convergent sequence in $S$ has a limit in $S$.


\newpage
\subsection{Completeness}
\subsection{Compactness}
\subsection{Connectedness}




% ------------------------------------------------------------------------------
% Reference and Cited Works
% ------------------------------------------------------------------------------

\bibliographystyle{IEEEtran}
\bibliography{References.bib}

% ------------------------------------------------------------------------------

\end{document}